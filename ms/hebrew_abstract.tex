%\title{Hebrew document in WriteLatex - מסמך בעברית}
\documentclass{article}
\usepackage[utf8x]{inputenc}
\usepackage[english,hebrew]{babel}
\selectlanguage{hebrew}
\usepackage{geometry}
\geometry{
a4paper,
total={170mm,257mm},
left=20mm,
top=20mm,
headheight=12pt
}
\begin{document}
\pagestyle{empty}

\section*{תקציר}
העתקה של מודל לחיקוי יכולה להיות שיטה יעילה לרכישת ידע. כאשר נדרשים לבחור מודל לחיקוי, ישנה הטייה ידועה הנקראת הטיה מבוססת הצלחה: כלומר, להעתיק מהמודל לחיקוי שנראה המצליח ביותר. הטיה זאת מבוססת על ביצועיו ויכולותיו של המודל לחיקוי בלבד, ללא גורמים נוספים. אנחנו מציעים הטייה נוספת שייתכן שהינה נפוצה בתהליכי הורשה תרבותית: הטייה מבוססת השפעה, כלומר, הבחירה במודל לחיקוי מושפעת מכמות המעתיקים שכבר בחרו מודל לחיקוי. במודל שלנו אנחנו משלבים את ההטיות לכדי הטייה אחת שאנו קוראים לה "הטיית יוקרה" ואנו מנתחים את ההשפעות של הטייה זו על הדינמיקות של אבולציה תרבותית, באמצעות שימוש באנליזה מתמטית וסימולציות סטוכסטיות.

במחקר זה מצאנו שיערוכים מתמטיים לתהליך הסטוכסטי שלנו, אשר סייעו לנו בפיתוח מתמטיים מורכבים יותר של המודל, ובייעול החישוביות של הסימולציות בסדר גודל. אנו מאששים שיערוכים אלה באמצעות סימולציות, ובודקים את חוסנם כנגד הנחות מקלות שבאמצעותן מצאנו שיערוכים אלו.
בנוסף, מצאנו שיערוכים ל"הסתברות קיבעון" )ההסתברות שאלמנט תרבותי יתפשט באוכלוסיה כולה( ו"זמן לקיבעון", בסביבות קבועות ומשתנות, אשר דומים לשיערוכי קימורה למודלים של אבולוציה גנטית.
שיערוכים אלו מראים שהטייה מבוססת הצלחה דומה לבחירה הטבעית, והטייה מבוססת השפעה מקטינה את גודל האוכלוסייה האפקטיבי.
הטייה מבוססת השפעה גם מאיצה את הדינמיקות של האבולוציה, כמו שניתן לצפות ממודל "עשירים מתעשרים". 

המודל שלנו יכול לתאר בצורה טובה תהליכי תורשה תרבותית, במיוחד בתרבויות של בני האדם, בהן רשתות המדיה החברתית פופלריות. עובדה נוספת נדרשת בכדי לבחון אם מודל זה יכול לחזות תופעות כלשהן באבולוציה תרבותית, כאשר מעשירים אותו בהשפעות של בחירה טבעית, ויוזמה אישית.

\pagebreak

עבודה זו בוצעה בהדרכתו של דר' יואב רם מבי"ס לזואולוגיה, הפקולטה למדעי החיים, אוניברסיטת תל אביב.



\end{document}